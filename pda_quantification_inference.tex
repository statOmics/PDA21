% Options for packages loaded elsewhere
\PassOptionsToPackage{unicode}{hyperref}
\PassOptionsToPackage{hyphens}{url}
%
\documentclass[
]{article}
\usepackage{amsmath,amssymb}
\usepackage{lmodern}
\usepackage{ifxetex,ifluatex}
\ifnum 0\ifxetex 1\fi\ifluatex 1\fi=0 % if pdftex
  \usepackage[T1]{fontenc}
  \usepackage[utf8]{inputenc}
  \usepackage{textcomp} % provide euro and other symbols
\else % if luatex or xetex
  \usepackage{unicode-math}
  \defaultfontfeatures{Scale=MatchLowercase}
  \defaultfontfeatures[\rmfamily]{Ligatures=TeX,Scale=1}
\fi
% Use upquote if available, for straight quotes in verbatim environments
\IfFileExists{upquote.sty}{\usepackage{upquote}}{}
\IfFileExists{microtype.sty}{% use microtype if available
  \usepackage[]{microtype}
  \UseMicrotypeSet[protrusion]{basicmath} % disable protrusion for tt fonts
}{}
\makeatletter
\@ifundefined{KOMAClassName}{% if non-KOMA class
  \IfFileExists{parskip.sty}{%
    \usepackage{parskip}
  }{% else
    \setlength{\parindent}{0pt}
    \setlength{\parskip}{6pt plus 2pt minus 1pt}}
}{% if KOMA class
  \KOMAoptions{parskip=half}}
\makeatother
\usepackage{xcolor}
\IfFileExists{xurl.sty}{\usepackage{xurl}}{} % add URL line breaks if available
\IfFileExists{bookmark.sty}{\usepackage{bookmark}}{\usepackage{hyperref}}
\hypersetup{
  pdftitle={Statistical Methods for Quantitative MS-based Proteomics: Part II. Differential Abundance Analysis},
  pdfauthor={Lieven Clement},
  hidelinks,
  pdfcreator={LaTeX via pandoc}}
\urlstyle{same} % disable monospaced font for URLs
\usepackage[margin=1in]{geometry}
\usepackage{color}
\usepackage{fancyvrb}
\newcommand{\VerbBar}{|}
\newcommand{\VERB}{\Verb[commandchars=\\\{\}]}
\DefineVerbatimEnvironment{Highlighting}{Verbatim}{commandchars=\\\{\}}
% Add ',fontsize=\small' for more characters per line
\usepackage{framed}
\definecolor{shadecolor}{RGB}{248,248,248}
\newenvironment{Shaded}{\begin{snugshade}}{\end{snugshade}}
\newcommand{\AlertTok}[1]{\textcolor[rgb]{0.94,0.16,0.16}{#1}}
\newcommand{\AnnotationTok}[1]{\textcolor[rgb]{0.56,0.35,0.01}{\textbf{\textit{#1}}}}
\newcommand{\AttributeTok}[1]{\textcolor[rgb]{0.77,0.63,0.00}{#1}}
\newcommand{\BaseNTok}[1]{\textcolor[rgb]{0.00,0.00,0.81}{#1}}
\newcommand{\BuiltInTok}[1]{#1}
\newcommand{\CharTok}[1]{\textcolor[rgb]{0.31,0.60,0.02}{#1}}
\newcommand{\CommentTok}[1]{\textcolor[rgb]{0.56,0.35,0.01}{\textit{#1}}}
\newcommand{\CommentVarTok}[1]{\textcolor[rgb]{0.56,0.35,0.01}{\textbf{\textit{#1}}}}
\newcommand{\ConstantTok}[1]{\textcolor[rgb]{0.00,0.00,0.00}{#1}}
\newcommand{\ControlFlowTok}[1]{\textcolor[rgb]{0.13,0.29,0.53}{\textbf{#1}}}
\newcommand{\DataTypeTok}[1]{\textcolor[rgb]{0.13,0.29,0.53}{#1}}
\newcommand{\DecValTok}[1]{\textcolor[rgb]{0.00,0.00,0.81}{#1}}
\newcommand{\DocumentationTok}[1]{\textcolor[rgb]{0.56,0.35,0.01}{\textbf{\textit{#1}}}}
\newcommand{\ErrorTok}[1]{\textcolor[rgb]{0.64,0.00,0.00}{\textbf{#1}}}
\newcommand{\ExtensionTok}[1]{#1}
\newcommand{\FloatTok}[1]{\textcolor[rgb]{0.00,0.00,0.81}{#1}}
\newcommand{\FunctionTok}[1]{\textcolor[rgb]{0.00,0.00,0.00}{#1}}
\newcommand{\ImportTok}[1]{#1}
\newcommand{\InformationTok}[1]{\textcolor[rgb]{0.56,0.35,0.01}{\textbf{\textit{#1}}}}
\newcommand{\KeywordTok}[1]{\textcolor[rgb]{0.13,0.29,0.53}{\textbf{#1}}}
\newcommand{\NormalTok}[1]{#1}
\newcommand{\OperatorTok}[1]{\textcolor[rgb]{0.81,0.36,0.00}{\textbf{#1}}}
\newcommand{\OtherTok}[1]{\textcolor[rgb]{0.56,0.35,0.01}{#1}}
\newcommand{\PreprocessorTok}[1]{\textcolor[rgb]{0.56,0.35,0.01}{\textit{#1}}}
\newcommand{\RegionMarkerTok}[1]{#1}
\newcommand{\SpecialCharTok}[1]{\textcolor[rgb]{0.00,0.00,0.00}{#1}}
\newcommand{\SpecialStringTok}[1]{\textcolor[rgb]{0.31,0.60,0.02}{#1}}
\newcommand{\StringTok}[1]{\textcolor[rgb]{0.31,0.60,0.02}{#1}}
\newcommand{\VariableTok}[1]{\textcolor[rgb]{0.00,0.00,0.00}{#1}}
\newcommand{\VerbatimStringTok}[1]{\textcolor[rgb]{0.31,0.60,0.02}{#1}}
\newcommand{\WarningTok}[1]{\textcolor[rgb]{0.56,0.35,0.01}{\textbf{\textit{#1}}}}
\usepackage{longtable,booktabs,array}
\usepackage{calc} % for calculating minipage widths
% Correct order of tables after \paragraph or \subparagraph
\usepackage{etoolbox}
\makeatletter
\patchcmd\longtable{\par}{\if@noskipsec\mbox{}\fi\par}{}{}
\makeatother
% Allow footnotes in longtable head/foot
\IfFileExists{footnotehyper.sty}{\usepackage{footnotehyper}}{\usepackage{footnote}}
\makesavenoteenv{longtable}
\usepackage{graphicx}
\makeatletter
\def\maxwidth{\ifdim\Gin@nat@width>\linewidth\linewidth\else\Gin@nat@width\fi}
\def\maxheight{\ifdim\Gin@nat@height>\textheight\textheight\else\Gin@nat@height\fi}
\makeatother
% Scale images if necessary, so that they will not overflow the page
% margins by default, and it is still possible to overwrite the defaults
% using explicit options in \includegraphics[width, height, ...]{}
\setkeys{Gin}{width=\maxwidth,height=\maxheight,keepaspectratio}
% Set default figure placement to htbp
\makeatletter
\def\fps@figure{htbp}
\makeatother
\setlength{\emergencystretch}{3em} % prevent overfull lines
\providecommand{\tightlist}{%
  \setlength{\itemsep}{0pt}\setlength{\parskip}{0pt}}
\setcounter{secnumdepth}{5}
\ifluatex
  \usepackage{selnolig}  % disable illegal ligatures
\fi

\title{Statistical Methods for Quantitative MS-based Proteomics: Part
II. Differential Abundance Analysis}
\author{Lieven Clement}
\date{June 23, 2021}

\begin{document}
\maketitle

{
\setcounter{tocdepth}{2}
\tableofcontents
}
\hypertarget{outline}{%
\section*{Outline}\label{outline}}
\addcontentsline{toc}{section}{Outline}

\begin{itemize}
\tightlist
\item
  Francisella tularensis Example
\item
  Hypothesis testing
\item
  Multiple testing
\item
  Moderated statistics
\item
  Experimental design
\end{itemize}

\begin{center}\rule{0.5\linewidth}{0.5pt}\end{center}

\hypertarget{francisella-tularensis-experiment}{%
\section{Francisella tularensis
experiment}\label{francisella-tularensis-experiment}}

\includegraphics{./figures/francisella.jpg}

\includegraphics{./figures/tularemia_lesion.jpg}

\begin{itemize}
\tightlist
\item
  Pathogen: causes tularemia
\item
  Metabolic adaptation key for intracellular life cycle of pathogenic
  microorganisms.
\item
  Upon entry into host cells quick phasomal escape and active
  multiplication in cytosolic compartment.
\item
  Franciscella is auxotroph for several amino acids, including arginine.
\item
  Inactivation of arginine transporter delayed bacterial phagosomal
  escape and intracellular multiplication.
\item
  Experiment to assess difference in proteome using 3 WT vs 3 ArgP KO
  mutants
\end{itemize}

\hypertarget{import-the-data-in-r}{%
\subsection{Import the data in R}\label{import-the-data-in-r}}

Click to see code

\begin{enumerate}
\def\labelenumi{\arabic{enumi}.}
\tightlist
\item
  Load libraries
\end{enumerate}

\begin{Shaded}
\begin{Highlighting}[]
\FunctionTok{library}\NormalTok{(tidyverse)}
\FunctionTok{library}\NormalTok{(limma)}
\FunctionTok{library}\NormalTok{(QFeatures)}
\FunctionTok{library}\NormalTok{(msqrob2)}
\FunctionTok{library}\NormalTok{(plotly)}
\FunctionTok{library}\NormalTok{(ggplot2)}
\end{Highlighting}
\end{Shaded}

\begin{enumerate}
\def\labelenumi{\arabic{enumi}.}
\setcounter{enumi}{1}
\tightlist
\item
  We use a peptides.txt file from MS-data quantified with maxquant that
  contains MS1 intensities summarized at the peptide level.
\end{enumerate}

\begin{Shaded}
\begin{Highlighting}[]
\NormalTok{peptidesFile }\OtherTok{\textless{}{-}} \StringTok{"https://raw.githubusercontent.com/statOmics/PDA20/data/quantification/francisella/peptides.txt"}
\end{Highlighting}
\end{Shaded}

\begin{enumerate}
\def\labelenumi{\arabic{enumi}.}
\setcounter{enumi}{2}
\tightlist
\item
  Maxquant stores the intensity data for the different samples in
  columnns that start with Intensity. We can retreive the column names
  with the intensity data with the code below:
\end{enumerate}

\begin{Shaded}
\begin{Highlighting}[]
\NormalTok{ecols }\OtherTok{\textless{}{-}} \FunctionTok{grep}\NormalTok{(}\StringTok{"Intensity}\SpecialCharTok{\textbackslash{}\textbackslash{}}\StringTok{."}\NormalTok{, }\FunctionTok{names}\NormalTok{(}\FunctionTok{read.delim}\NormalTok{(peptidesFile)))}
\end{Highlighting}
\end{Shaded}

\begin{enumerate}
\def\labelenumi{\arabic{enumi}.}
\setcounter{enumi}{3}
\tightlist
\item
  Read the data and store it in QFeatures object
\end{enumerate}

\begin{Shaded}
\begin{Highlighting}[]
\NormalTok{pe }\OtherTok{\textless{}{-}} \FunctionTok{readQFeatures}\NormalTok{(}
  \AttributeTok{table =}\NormalTok{ peptidesFile,}
  \AttributeTok{fnames =} \DecValTok{1}\NormalTok{,}
  \AttributeTok{ecol =}\NormalTok{ ecols,}
  \AttributeTok{name =} \StringTok{"peptideRaw"}\NormalTok{, }\AttributeTok{sep=}\StringTok{"}\SpecialCharTok{\textbackslash{}t}\StringTok{"}\NormalTok{)}
\end{Highlighting}
\end{Shaded}

\begin{enumerate}
\def\labelenumi{\arabic{enumi}.}
\setcounter{enumi}{4}
\tightlist
\item
  Update data with information on design
\end{enumerate}

\begin{Shaded}
\begin{Highlighting}[]
\FunctionTok{colData}\NormalTok{(pe)}\SpecialCharTok{$}\NormalTok{genotype }\OtherTok{\textless{}{-}}\NormalTok{ pe[[}\DecValTok{1}\NormalTok{]] }\SpecialCharTok{\%\textgreater{}\%} 
\NormalTok{  colnames }\SpecialCharTok{\%\textgreater{}\%} 
  \FunctionTok{substr}\NormalTok{(}\DecValTok{12}\NormalTok{,}\DecValTok{13}\NormalTok{) }\SpecialCharTok{\%\textgreater{}\%}
\NormalTok{  as.factor }\SpecialCharTok{\%\textgreater{}\%} 
  \FunctionTok{relevel}\NormalTok{(}\StringTok{"WT"}\NormalTok{)}
\NormalTok{pe }\SpecialCharTok{\%\textgreater{}\%}\NormalTok{ colData}
\end{Highlighting}
\end{Shaded}

\begin{verbatim}
## DataFrame with 6 rows and 1 column
##                          genotype
##                          <factor>
## Intensity.1WT_20_2h_n3_3       WT
## Intensity.1WT_20_2h_n4_3       WT
## Intensity.1WT_20_2h_n5_3       WT
## Intensity.3D8_20_2h_n3_3       D8
## Intensity.3D8_20_2h_n4_3       D8
## Intensity.3D8_20_2h_n5_3       D8
\end{verbatim}

\hypertarget{preprocessing}{%
\subsection{Preprocessing}\label{preprocessing}}

Click to see code to log-transfrom the data

\begin{enumerate}
\def\labelenumi{\arabic{enumi}.}
\tightlist
\item
  Log transform
\end{enumerate}

\begin{itemize}
\tightlist
\item
  Calculate number of non zero intensities for each peptide
\end{itemize}

\begin{Shaded}
\begin{Highlighting}[]
\FunctionTok{rowData}\NormalTok{(pe[[}\StringTok{"peptideRaw"}\NormalTok{]])}\SpecialCharTok{$}\NormalTok{nNonZero }\OtherTok{\textless{}{-}} \FunctionTok{rowSums}\NormalTok{(}\FunctionTok{assay}\NormalTok{(pe[[}\StringTok{"peptideRaw"}\NormalTok{]]) }\SpecialCharTok{\textgreater{}} \DecValTok{0}\NormalTok{)}
\end{Highlighting}
\end{Shaded}

\begin{itemize}
\tightlist
\item
  Peptides with zero intensities are missing peptides and should be
  represent with a \texttt{NA} value rather than \texttt{0}.
\end{itemize}

\begin{Shaded}
\begin{Highlighting}[]
\NormalTok{pe }\OtherTok{\textless{}{-}} \FunctionTok{zeroIsNA}\NormalTok{(pe, }\StringTok{"peptideRaw"}\NormalTok{) }\CommentTok{\# convert 0 to NA}
\end{Highlighting}
\end{Shaded}

\begin{itemize}
\tightlist
\item
  Logtransform data with base 2
\end{itemize}

\begin{Shaded}
\begin{Highlighting}[]
\NormalTok{pe }\OtherTok{\textless{}{-}} \FunctionTok{logTransform}\NormalTok{(pe, }\AttributeTok{base =} \DecValTok{2}\NormalTok{, }\AttributeTok{i =} \StringTok{"peptideRaw"}\NormalTok{, }\AttributeTok{name =} \StringTok{"peptideLog"}\NormalTok{)}
\end{Highlighting}
\end{Shaded}

\begin{enumerate}
\def\labelenumi{\arabic{enumi}.}
\setcounter{enumi}{1}
\tightlist
\item
  Filtering
\end{enumerate}

\begin{itemize}
\tightlist
\item
  Handling overlapping protein groups
\end{itemize}

\begin{Shaded}
\begin{Highlighting}[]
\NormalTok{pe[[}\StringTok{"peptideLog"}\NormalTok{]] }\OtherTok{\textless{}{-}}
\NormalTok{ pe[[}\StringTok{"peptideLog"}\NormalTok{]][}\FunctionTok{rowData}\NormalTok{(pe[[}\StringTok{"peptideLog"}\NormalTok{]])}\SpecialCharTok{$}\NormalTok{Proteins}
 \SpecialCharTok{\%in\%} \FunctionTok{smallestUniqueGroups}\NormalTok{(}\FunctionTok{rowData}\NormalTok{(pe[[}\StringTok{"peptideLog"}\NormalTok{]])}\SpecialCharTok{$}\NormalTok{Proteins),]}
\end{Highlighting}
\end{Shaded}

\begin{itemize}
\tightlist
\item
  Remove reverse sequences (decoys) and contaminants. Note that this is
  indicated by the column names Reverse and depending on the version of
  maxQuant with Potential.contaminants or Contaminants.
\end{itemize}

\begin{Shaded}
\begin{Highlighting}[]
\NormalTok{pe[[}\StringTok{"peptideLog"}\NormalTok{]] }\OtherTok{\textless{}{-}}\NormalTok{ pe[[}\StringTok{"peptideLog"}\NormalTok{]][}\FunctionTok{rowData}\NormalTok{(pe[[}\StringTok{"peptideLog"}\NormalTok{]])}\SpecialCharTok{$}\NormalTok{Reverse }\SpecialCharTok{!=} \StringTok{"+"}\NormalTok{, ]}
\NormalTok{pe[[}\StringTok{"peptideLog"}\NormalTok{]] }\OtherTok{\textless{}{-}}\NormalTok{ pe[[}\StringTok{"peptideLog"}\NormalTok{]][}\FunctionTok{rowData}\NormalTok{(pe[[}\StringTok{"peptideLog"}\NormalTok{]])}\SpecialCharTok{$}
\NormalTok{    Contaminant }\SpecialCharTok{!=} \StringTok{"+"}\NormalTok{, ]}
\end{Highlighting}
\end{Shaded}

\begin{itemize}
\tightlist
\item
  Drop peptides that were only identified in one sample
\end{itemize}

\begin{Shaded}
\begin{Highlighting}[]
\NormalTok{pe[[}\StringTok{"peptideLog"}\NormalTok{]] }\OtherTok{\textless{}{-}}\NormalTok{ pe[[}\StringTok{"peptideLog"}\NormalTok{]][}\FunctionTok{rowData}\NormalTok{(pe[[}\StringTok{"peptideLog"}\NormalTok{]])}\SpecialCharTok{$}\NormalTok{nNonZero }\SpecialCharTok{\textgreater{}=} \DecValTok{2}\NormalTok{, ]}
\FunctionTok{nrow}\NormalTok{(pe[[}\StringTok{"peptideLog"}\NormalTok{]])}
\end{Highlighting}
\end{Shaded}

\begin{verbatim}
## [1] 6525
\end{verbatim}

We keep 6525 peptides upon filtering.

\begin{enumerate}
\def\labelenumi{\arabic{enumi}.}
\setcounter{enumi}{2}
\tightlist
\item
  Normalization by median centering
\end{enumerate}

\begin{Shaded}
\begin{Highlighting}[]
\NormalTok{pe }\OtherTok{\textless{}{-}} \FunctionTok{normalize}\NormalTok{(pe, }
                \AttributeTok{i =} \StringTok{"peptideLog"}\NormalTok{, }
                \AttributeTok{name =} \StringTok{"peptideNorm"}\NormalTok{, }
                \AttributeTok{method =} \StringTok{"center.median"}\NormalTok{)}
\end{Highlighting}
\end{Shaded}

\begin{enumerate}
\def\labelenumi{\arabic{enumi}.}
\setcounter{enumi}{3}
\tightlist
\item
  Summarization. We use the standard sumarisation in aggregateFeatures,
  which is a robust summarisation method.
\end{enumerate}

\begin{Shaded}
\begin{Highlighting}[]
\NormalTok{pe }\OtherTok{\textless{}{-}} \FunctionTok{aggregateFeatures}\NormalTok{(pe,}
    \AttributeTok{i =} \StringTok{"peptideNorm"}\NormalTok{, }
    \AttributeTok{fcol =} \StringTok{"Proteins"}\NormalTok{, }
    \AttributeTok{na.rm =} \ConstantTok{TRUE}\NormalTok{,}
    \AttributeTok{name =} \StringTok{"protein"}\NormalTok{)}
\end{Highlighting}
\end{Shaded}

\begin{verbatim}
## Your quantitative and row data contain missing values. Please read the
## relevant section(s) in the aggregateFeatures manual page regarding the
## effects of missing values on data aggregation.
\end{verbatim}

Plot of preprocessed data

\begin{Shaded}
\begin{Highlighting}[]
\NormalTok{pe[[}\StringTok{"peptideNorm"}\NormalTok{]] }\SpecialCharTok{\%\textgreater{}\%} 
\NormalTok{  assay }\SpecialCharTok{\%\textgreater{}\%}
  \FunctionTok{as.data.frame}\NormalTok{() }\SpecialCharTok{\%\textgreater{}\%}
  \FunctionTok{gather}\NormalTok{(sample, intensity) }\SpecialCharTok{\%\textgreater{}\%} 
  \FunctionTok{mutate}\NormalTok{(}\AttributeTok{genotype =} \FunctionTok{colData}\NormalTok{(pe)[sample,}\StringTok{"genotype"}\NormalTok{]) }\SpecialCharTok{\%\textgreater{}\%}
  \FunctionTok{ggplot}\NormalTok{(}\FunctionTok{aes}\NormalTok{(}\AttributeTok{x =}\NormalTok{ intensity,}\AttributeTok{group =}\NormalTok{ sample,}\AttributeTok{color =}\NormalTok{ genotype)) }\SpecialCharTok{+} 
    \FunctionTok{geom\_density}\NormalTok{() }\SpecialCharTok{+}
    \FunctionTok{ggtitle}\NormalTok{(}\StringTok{"Peptide{-}level"}\NormalTok{)}
\end{Highlighting}
\end{Shaded}

\begin{verbatim}
## Warning: Removed 7561 rows containing non-finite values (stat_density).
\end{verbatim}

\includegraphics{pda_quantification_inference_files/figure-latex/unnamed-chunk-16-1.pdf}

\begin{Shaded}
\begin{Highlighting}[]
\NormalTok{pe[[}\StringTok{"protein"}\NormalTok{]] }\SpecialCharTok{\%\textgreater{}\%} 
\NormalTok{  assay }\SpecialCharTok{\%\textgreater{}\%}
  \FunctionTok{as.data.frame}\NormalTok{() }\SpecialCharTok{\%\textgreater{}\%}
  \FunctionTok{gather}\NormalTok{(sample, intensity) }\SpecialCharTok{\%\textgreater{}\%} 
  \FunctionTok{mutate}\NormalTok{(}\AttributeTok{genotype =} \FunctionTok{colData}\NormalTok{(pe)[sample,}\StringTok{"genotype"}\NormalTok{]) }\SpecialCharTok{\%\textgreater{}\%}
  \FunctionTok{ggplot}\NormalTok{(}\FunctionTok{aes}\NormalTok{(}\AttributeTok{x =}\NormalTok{ intensity,}\AttributeTok{group =}\NormalTok{ sample,}\AttributeTok{color =}\NormalTok{ genotype)) }\SpecialCharTok{+} 
    \FunctionTok{geom\_density}\NormalTok{() }\SpecialCharTok{+}
    \FunctionTok{ggtitle}\NormalTok{(}\StringTok{"Protein{-}level"}\NormalTok{)}
\end{Highlighting}
\end{Shaded}

\begin{verbatim}
## Warning: Removed 428 rows containing non-finite values (stat_density).
\end{verbatim}

\includegraphics{pda_quantification_inference_files/figure-latex/unnamed-chunk-16-2.pdf}

\hypertarget{summarized-data-structure}{%
\subsection{Summarized data structure}\label{summarized-data-structure}}

\hypertarget{design}{%
\subsubsection{Design}\label{design}}

\begin{Shaded}
\begin{Highlighting}[]
\NormalTok{pe }\SpecialCharTok{\%\textgreater{}\%} 
\NormalTok{  colData }\SpecialCharTok{\%\textgreater{}\%} 
\NormalTok{  knitr}\SpecialCharTok{::}\FunctionTok{kable}\NormalTok{()}
\end{Highlighting}
\end{Shaded}

\begin{longtable}[]{@{}ll@{}}
\toprule
& genotype \\
\midrule
\endhead
Intensity.1WT\_20\_2h\_n3\_3 & WT \\
Intensity.1WT\_20\_2h\_n4\_3 & WT \\
Intensity.1WT\_20\_2h\_n5\_3 & WT \\
Intensity.3D8\_20\_2h\_n3\_3 & D8 \\
Intensity.3D8\_20\_2h\_n4\_3 & D8 \\
Intensity.3D8\_20\_2h\_n5\_3 & D8 \\
\bottomrule
\end{longtable}

\begin{itemize}
\tightlist
\item
  WT vs KO
\item
  3 vs 3 repeats
\end{itemize}

\hypertarget{summarized-intensity-matrix}{%
\subsubsection{Summarized intensity
matrix}\label{summarized-intensity-matrix}}

\begin{Shaded}
\begin{Highlighting}[]
\NormalTok{pe[[}\StringTok{"protein"}\NormalTok{]] }\SpecialCharTok{\%\textgreater{}\%} \FunctionTok{assay}\NormalTok{() }\SpecialCharTok{\%\textgreater{}\%} \FunctionTok{head}\NormalTok{() }\SpecialCharTok{\%\textgreater{}\%}\NormalTok{ knitr}\SpecialCharTok{::}\FunctionTok{kable}\NormalTok{()}
\end{Highlighting}
\end{Shaded}

\begin{longtable}[]{@{}
  >{\raggedright\arraybackslash}p{(\columnwidth - 12\tabcolsep) * \real{0.08}}
  >{\raggedleft\arraybackslash}p{(\columnwidth - 12\tabcolsep) * \real{0.15}}
  >{\raggedleft\arraybackslash}p{(\columnwidth - 12\tabcolsep) * \real{0.15}}
  >{\raggedleft\arraybackslash}p{(\columnwidth - 12\tabcolsep) * \real{0.15}}
  >{\raggedleft\arraybackslash}p{(\columnwidth - 12\tabcolsep) * \real{0.15}}
  >{\raggedleft\arraybackslash}p{(\columnwidth - 12\tabcolsep) * \real{0.15}}
  >{\raggedleft\arraybackslash}p{(\columnwidth - 12\tabcolsep) * \real{0.15}}@{}}
\toprule
& Intensity.1WT\_20\_2h\_n3\_3 & Intensity.1WT\_20\_2h\_n4\_3 &
Intensity.1WT\_20\_2h\_n5\_3 & Intensity.3D8\_20\_2h\_n3\_3 &
Intensity.3D8\_20\_2h\_n4\_3 & Intensity.3D8\_20\_2h\_n5\_3 \\
\midrule
\endhead
WP\_003013731 & -0.2748775 & -0.0856247 & 0.1595370 & -0.2809009 &
0.0035526 & 0.0567110 \\
WP\_003013860 & NA & NA & -0.2512039 & NA & NA & -0.4865646 \\
WP\_003013909 & -0.6851118 & -0.8161658 & -0.7557906 & -0.4591476 &
-0.5449424 & -0.4962482 \\
WP\_003014068 & 0.6495386 & 0.8522239 & 1.1344852 & 0.5459176 &
0.9187714 & 0.5974741 \\
WP\_003014122 & -0.7630863 & -1.0430741 & -0.8091715 & -1.1743951 &
-1.1924725 & -1.2565893 \\
WP\_003014123 & -0.2051672 & -0.3361704 & -0.2151930 & -0.3855747 &
-0.2802011 & -0.5801771 \\
\bottomrule
\end{longtable}

\begin{itemize}
\tightlist
\item
  1115 proteins
\end{itemize}

\hypertarget{hypothesis-testing-a-single-protein}{%
\subsubsection{Hypothesis testing: a single
protein}\label{hypothesis-testing-a-single-protein}}

\includegraphics{pda_quantification_inference_files/figure-latex/unnamed-chunk-19-1.pdf}

\hypertarget{t-test}{%
\paragraph{T-test}\label{t-test}}

\[
 \log_2 \text{FC} = \bar{y}_{p1}-\bar{y}_{p2}
\]

\[
T_g=\frac{\log_2 \text{FC}}{\text{se}_{\log_2 \text{FC}}}
\]

\[
T_g=\frac{\widehat{\text{signal}}}{\widehat{\text{Noise}}}
\]

If we can assume equal variance in both treatment groups:

\[
\text{se}_{\log_2 \text{FC}}=\text{SD}\sqrt{\frac{1}{n_1}+\frac{1}{n_2}}
\]

\begin{Shaded}
\begin{Highlighting}[]
\NormalTok{WP\_003023392 }\OtherTok{\textless{}{-}} \FunctionTok{data.frame}\NormalTok{(}
    \AttributeTok{intensity =} \FunctionTok{assay}\NormalTok{(pe[[}\StringTok{"protein"}\NormalTok{]][}\StringTok{"WP\_003023392"}\NormalTok{,]) }\SpecialCharTok{\%\textgreater{}\%} \FunctionTok{c}\NormalTok{(), }
    \AttributeTok{genotype =} \FunctionTok{colData}\NormalTok{(pe)[,}\DecValTok{1}\NormalTok{]) }

\NormalTok{WP\_003023392 }\SpecialCharTok{\%\textgreater{}\%} 
  \FunctionTok{ggplot}\NormalTok{(}\FunctionTok{aes}\NormalTok{(}\AttributeTok{x=}\NormalTok{genotype,}\AttributeTok{y=}\NormalTok{intensity)) }\SpecialCharTok{+} 
  \FunctionTok{geom\_point}\NormalTok{() }\SpecialCharTok{+}
  \FunctionTok{ggtitle}\NormalTok{(}\StringTok{"Protein WP\_003023392"}\NormalTok{)}
\end{Highlighting}
\end{Shaded}

\includegraphics{pda_quantification_inference_files/figure-latex/unnamed-chunk-20-1.pdf}

\[
t=\frac{\log_2\widehat{\text{FC}}}{\text{se}_{\log_2\widehat{\text{FC}}}}=\frac{-1.43}{0.0577}=-24.7
\]

\begin{itemize}
\item
  Is t = -24.7 indicating that there is an effect?
\item
  How likely is it to observe t = -24.7 when there is no effect of the
  argP KO on the protein expression?
\end{itemize}

\hypertarget{null-hypothesis-h_0-and-alternative-hypothesis-h_1}{%
\paragraph{\texorpdfstring{Null hypothesis (\(H_0\)) and alternative
hypothesis
(\(H_1\))}{Null hypothesis (H\_0) and alternative hypothesis (H\_1)}}\label{null-hypothesis-h_0-and-alternative-hypothesis-h_1}}

\begin{itemize}
\item
  With data we can never prove a hypothesis (falsification principle of
  Popper)
\item
  With data we can only reject a hypothesis
\item
  In general we start from \emph{alternative hypothese} \(H_1\): we want
  to show an effect of the KO on a protein
\end{itemize}

\(H_1\): On average the protein abundance in WT is different from that
in KO

\begin{itemize}
\tightlist
\item
  But, we will assess this by falsifying the opposite:

  \(H_0\): On average the protein abundance in WT is equal to that in
  KO\textless-
\end{itemize}

\begin{Shaded}
\begin{Highlighting}[]
\FunctionTok{t.test}\NormalTok{(intensity }\SpecialCharTok{\textasciitilde{}}\NormalTok{ genotype, }\AttributeTok{data =}\NormalTok{ WP\_003023392)}
\end{Highlighting}
\end{Shaded}

\begin{verbatim}
## 
##  Welch Two Sample t-test
## 
## data:  intensity by genotype
## t = 24.747, df = 3.1653, p-value = 9.911e-05
## alternative hypothesis: true difference in means between group WT and group D8 is not equal to 0
## 95 percent confidence interval:
##  1.249550 1.606175
## sample estimates:
## mean in group WT mean in group D8 
##       -0.1821147       -1.6099769
\end{verbatim}

\begin{itemize}
\item
  How likely is it to observe an equal or more extreme effect than the
  one observed in the sample when the null hypothesis is true?
\item
  When we make assumptions about the distribution of our test statistic
  we can quantify this probability: \emph{p-value}. The p-value will
  only be calculated correctly if the underlying assumptions hold!
\item
  When we repeat the experiment, the probability to observe a fold
  change more extreme than a 0.372 fold (\(\log_2 FC=-1.43\)) down or up
  regulation by random change (if \(H_0\) is true) is 3 out of
  100.000.\\
\item
  If the p-value is below a significance threshold \(\alpha\) we reject
  the null hypothesis. \emph{We control the probability on a false
  positive result at the \(\alpha\)-level (type I error)}
\item
  Note, that the p-values are uniform under the null hypothesis,
  i.e.~when \(H_0\) is true all p-values are equally likely.
\end{itemize}

\hypertarget{multiple-hypothesis-testing}{%
\subsection{Multiple hypothesis
testing}\label{multiple-hypothesis-testing}}

\begin{itemize}
\item
  Consider testing DA for all \(m=1115\) proteins simultaneously
\item
  What if we assess each individual test at level \(\alpha\)?
  \(\rightarrow\) Probability to have a false positive among all m
  simultatenous test \(>>> \alpha= 0.05\)
\item
  Suppose that 800 proteins are non-DA, then we could expect to discover
  on average 800 × 0.05 = 40 false positive proteins. Hence, we are
  bound to call false positive proteins each time we run the experiment.
\end{itemize}

\hypertarget{multiple-testing}{%
\subsubsection{Multiple testing}\label{multiple-testing}}

When we want to infer on differential abundance of multiple proteins we
have to address the multiple testing issue.

\hypertarget{family-wise-error-rate}{%
\paragraph{Family-wise error rate}\label{family-wise-error-rate}}

The family-wise error rate (FWER) addresses the multiple testing issue
by no longer controlling the individual type I error for each protein,
instead it controls:

\[
   \text{FWER}=\text{P}\left[\text{reject at least one }H_{0i} \mid H_0\text{ is true}\right].
\]

The Bonferroni method is widely used to control the type I error:

\begin{itemize}
\tightlist
\item
  assess each test at \[\alpha_\text{adj}=\frac{\alpha}{m}\]
\item
  or use adjusted p-values and compare them to \(\alpha\):
  \[p_\text{adj}=\text{min}\left(p \times m,1\right)\]
\end{itemize}

Problem, the method is very conservative!

\hypertarget{false-discovery-rate}{%
\paragraph{False discovery rate}\label{false-discovery-rate}}

\begin{itemize}
\tightlist
\item
  FDR: Expected proportion of false positives on the total number of
  positives you return.
\item
  An FDR of 1\% means that on average we expect 1\% false positive
  proteins in the list of proteins that are called significant.
\item
  Defined by Benjamini and Hochberg in their seminal paper Benjamini, Y.
  and Hochberg, Y. (1995). ``Controlling the false discovery rate: a
  practical and powerful approach to multiple testing''. Journal of the
  Royal Statistical Society Series B, 57 (1): 289--300.
\end{itemize}

The table shows the results of \(m\) hypothesis tests in a single
experiment.

\begin{longtable}[]{@{}llll@{}}
\toprule
& accept \(H_{0i}\) & reject \(H_{0i}\) & Total \\
\midrule
\endhead
null & TN & FP & \(m_0\) \\
non-null & FN & TP & \(m_1\) \\
Total & NR & R & m \\
\bottomrule
\end{longtable}

\begin{itemize}
\tightlist
\item
  \(TN\): number of true negative: random and unobserved
\item
  \(FP\): number of false positives: random and unobserved
\item
  \(FN\): number of false negatives: random and unobserved
\item
  \(TP\): number of true positives: random and unobserved
\item
  \(NR\): number of acceptances (negative results): random and observed
\item
  \(R\): number of rejections (positive results): random and observed
\item
  \(m_0\) and \(m_1\): fixed and unobserved
\item
  \(m\): fixed and observed
\end{itemize}

The \textbf{False Discovery Proportion (FDP)} is the fraction of false
positives that are returned, i.e.~

\[
FDP = \frac{FP}{R}
\] However, this quantity cannot be observed because in practice we only
know \(R\) but we do not know \(FP\).

Therefore, Benjamini and Hochberg, 1995, defined The \textbf{False
Discovery Rate (FDR)} as \[
   \text{FDR} = \text{E}\left[\frac{FP}{R}\right] =\text{E}\left[\text{FDP}\right]
\] the expected FDP.

\begin{itemize}
\tightlist
\item
  Controlling the FDR allows for more discoveries (i.e.~longer lists
  with significant results), while the fraction of false discoveries
  among the significant results in well controlled on average. As a
  consequence, more of the true positive hypotheses will be detected.
\end{itemize}

The Benjamini and Hochberg (1995) procedure for controlling the FDR at
\(\alpha\):

\begin{enumerate}
\def\labelenumi{\arabic{enumi}.}
\item
  Let \(p_{(1)}\leq \ldots \leq p_{(m)}\) denote the ordered
  \(p\)-values.
\item
  Let \(k=\max\{i: p_{(i)}\leq i \alpha/m\}\), i.e.~\(k\) is the largest
  integer so that \(p_{(k)}\leq k \alpha/m\).
\item
  If such a \(k\) exists, reject the \(k\) null hypotheses associated
  with \(p_{(1)}, \ldots, p_{(k)}\). Otherwise none of the null
  hypotheses is rejected.
\end{enumerate}

The adjusted \(p\)-value (also known as the \(q\)-value in FDR
literature): \[
   q_{(i)}=\tilde{p}_{(i)} = \min\left[\min_{j=i,\ldots, m}\left(m p_{(j)}/j\right), 1 \right].
 \]

Click to see code

\begin{Shaded}
\begin{Highlighting}[]
\NormalTok{ttestMx }\OtherTok{\textless{}{-}} \ControlFlowTok{function}\NormalTok{(y,group) \{}
\NormalTok{    test }\OtherTok{\textless{}{-}} \FunctionTok{try}\NormalTok{(}\FunctionTok{t.test}\NormalTok{(y[group],y[}\SpecialCharTok{!}\NormalTok{group],}\AttributeTok{var.equal=}\ConstantTok{TRUE}\NormalTok{),}\AttributeTok{silent=}\ConstantTok{TRUE}\NormalTok{)}
    \ControlFlowTok{if}\NormalTok{(}\FunctionTok{is}\NormalTok{(test,}\StringTok{"try{-}error"}\NormalTok{)) \{}
      \FunctionTok{return}\NormalTok{(}\FunctionTok{c}\NormalTok{(}\AttributeTok{log2FC=}\ConstantTok{NA}\NormalTok{,}\AttributeTok{se=}\ConstantTok{NA}\NormalTok{,}\AttributeTok{tstat=}\ConstantTok{NA}\NormalTok{,}\AttributeTok{p=}\ConstantTok{NA}\NormalTok{))}
\NormalTok{      \} }\ControlFlowTok{else}\NormalTok{ \{}
      \FunctionTok{return}\NormalTok{(}\FunctionTok{c}\NormalTok{(}\AttributeTok{log2FC=}\NormalTok{ (test}\SpecialCharTok{$}\NormalTok{estimate}\SpecialCharTok{\%*\%}\FunctionTok{c}\NormalTok{(}\DecValTok{1}\NormalTok{,}\SpecialCharTok{{-}}\DecValTok{1}\NormalTok{)),}\AttributeTok{se=}\NormalTok{test}\SpecialCharTok{$}\NormalTok{stderr,}\AttributeTok{tstat=}\NormalTok{test}\SpecialCharTok{$}\NormalTok{statistic,}\AttributeTok{pval=}\NormalTok{test}\SpecialCharTok{$}\NormalTok{p.value))}
\NormalTok{      \}}
\NormalTok{ \}}
 
\NormalTok{ res }\OtherTok{\textless{}{-}} \FunctionTok{apply}\NormalTok{(}
    \FunctionTok{assay}\NormalTok{(pe[[}\StringTok{"protein"}\NormalTok{]]), }
    \DecValTok{1}\NormalTok{, }
\NormalTok{    ttestMx,}
    \AttributeTok{group =} \FunctionTok{colData}\NormalTok{(pe)}\SpecialCharTok{$}\NormalTok{genotype}\SpecialCharTok{==}\StringTok{"D8"}\NormalTok{) }\SpecialCharTok{\%\textgreater{}\%} 
\NormalTok{  t }
 \FunctionTok{colnames}\NormalTok{(res) }\OtherTok{\textless{}{-}} \FunctionTok{c}\NormalTok{(}\StringTok{"logFC"}\NormalTok{,}\StringTok{"se"}\NormalTok{,}\StringTok{"tstat"}\NormalTok{,}\StringTok{"pval"}\NormalTok{)}
\NormalTok{ res }\OtherTok{\textless{}{-}}\NormalTok{ res }\SpecialCharTok{\%\textgreater{}\%}\NormalTok{ as.data.frame }\SpecialCharTok{\%\textgreater{}\%}\NormalTok{ na.exclude }\SpecialCharTok{\%\textgreater{}\%} \FunctionTok{arrange}\NormalTok{(pval)}
\NormalTok{ res}\SpecialCharTok{$}\NormalTok{adjPval }\OtherTok{\textless{}{-}} \FunctionTok{p.adjust}\NormalTok{(res}\SpecialCharTok{$}\NormalTok{pval, }\StringTok{"fdr"}\NormalTok{)}
\NormalTok{ alpha }\OtherTok{\textless{}{-}} \FloatTok{0.05}
\NormalTok{res}\SpecialCharTok{$}\NormalTok{adjAlphaForm }\OtherTok{\textless{}{-}} \FunctionTok{paste0}\NormalTok{(}\DecValTok{1}\SpecialCharTok{:}\FunctionTok{nrow}\NormalTok{(res),}\StringTok{" x 0.05/"}\NormalTok{,}\FunctionTok{nrow}\NormalTok{(res))}
\NormalTok{res}\SpecialCharTok{$}\NormalTok{adjAlpha }\OtherTok{\textless{}{-}}\NormalTok{ alpha }\SpecialCharTok{*}\NormalTok{ (}\DecValTok{1}\SpecialCharTok{:}\FunctionTok{nrow}\NormalTok{(res))}\SpecialCharTok{/}\FunctionTok{nrow}\NormalTok{(res) }
\NormalTok{res}\SpecialCharTok{$}\StringTok{"pval \textless{} adjAlpha"} \OtherTok{\textless{}{-}}\NormalTok{ res}\SpecialCharTok{$}\NormalTok{pval }\SpecialCharTok{\textless{}}\NormalTok{ res}\SpecialCharTok{$}\NormalTok{adjAlpha }
\NormalTok{res}\SpecialCharTok{$}\StringTok{"adjPval \textless{} alpha"} \OtherTok{\textless{}{-}}\NormalTok{ res}\SpecialCharTok{$}\NormalTok{adjPval }\SpecialCharTok{\textless{}}\NormalTok{ alpha }
\end{Highlighting}
\end{Shaded}

\begin{longtable}[]{@{}lrrrlrll@{}}
\toprule
& logFC & pval & adjPval & adjAlphaForm & adjAlpha & pval \textless{}
adjAlpha & adjPval \textless{} alpha \\
\midrule
\endhead
WP\_003038940 & -0.2876290 & 0.0000146 & 0.0084347 & 1 x 0.05/1066 &
0.0000469 & TRUE & TRUE \\
WP\_003023392 & -1.4278622 & 0.0000158 & 0.0084347 & 2 x 0.05/1066 &
0.0000938 & TRUE & TRUE \\
WP\_003039212 & -0.2658247 & 0.0000820 & 0.0291520 & 3 x 0.05/1066 &
0.0001407 & TRUE & TRUE \\
WP\_003026016 & -1.0800305 & 0.0001395 & 0.0346124 & 4 x 0.05/1066 &
0.0001876 & TRUE & TRUE \\
WP\_003039615 & -0.3992190 & 0.0001623 & 0.0346124 & 5 x 0.05/1066 &
0.0002345 & TRUE & TRUE \\
WP\_011733588 & -0.4323262 & 0.0002291 & 0.0407034 & 6 x 0.05/1066 &
0.0002814 & TRUE & TRUE \\
WP\_003014552 & -0.9843865 & 0.0003224 & 0.0440266 & 7 x 0.05/1066 &
0.0003283 & TRUE & TRUE \\
WP\_003040849 & -1.2780743 & 0.0003304 & 0.0440266 & 8 x 0.05/1066 &
0.0003752 & TRUE & TRUE \\
WP\_003038430 & -0.4331987 & 0.0004505 & 0.0489078 & 9 x 0.05/1066 &
0.0004221 & FALSE & TRUE \\
WP\_003033975 & -0.2949061 & 0.0005047 & 0.0489078 & 10 x 0.05/1066 &
0.0004690 & FALSE & TRUE \\
\bottomrule
\end{longtable}

Click to see code

\begin{Shaded}
\begin{Highlighting}[]
\NormalTok{volcanoT }\OtherTok{\textless{}{-}}\NormalTok{ res }\SpecialCharTok{\%\textgreater{}\%} 
  \FunctionTok{ggplot}\NormalTok{(}\FunctionTok{aes}\NormalTok{(}\AttributeTok{x =}\NormalTok{ logFC, }\AttributeTok{y =} \SpecialCharTok{{-}}\FunctionTok{log10}\NormalTok{(pval), }\AttributeTok{color =}\NormalTok{ adjPval }\SpecialCharTok{\textless{}} \FloatTok{0.05}\NormalTok{)) }\SpecialCharTok{+}
    \FunctionTok{geom\_point}\NormalTok{(}\AttributeTok{cex =} \FloatTok{2.5}\NormalTok{) }\SpecialCharTok{+}
    \FunctionTok{scale\_color\_manual}\NormalTok{(}\AttributeTok{values =} \FunctionTok{alpha}\NormalTok{(}\FunctionTok{c}\NormalTok{(}\StringTok{"black"}\NormalTok{, }\StringTok{"red"}\NormalTok{), }\FloatTok{0.5}\NormalTok{)) }\SpecialCharTok{+}
    \FunctionTok{theme\_minimal}\NormalTok{() }
\end{Highlighting}
\end{Shaded}

\begin{Shaded}
\begin{Highlighting}[]
\NormalTok{volcanoT}
\end{Highlighting}
\end{Shaded}

\includegraphics{pda_quantification_inference_files/figure-latex/unnamed-chunk-26-1.pdf}

\hypertarget{moderated-statistics}{%
\subsection{Moderated Statistics}\label{moderated-statistics}}

Problems with ordinary t-test

\begin{Shaded}
\begin{Highlighting}[]
\NormalTok{res }\SpecialCharTok{\%\textgreater{}\%} 
  \FunctionTok{ggplot}\NormalTok{(}\FunctionTok{aes}\NormalTok{(}\AttributeTok{x =}\NormalTok{ logFC, }\AttributeTok{y =}\NormalTok{ se, }\AttributeTok{color =}\NormalTok{ adjPval }\SpecialCharTok{\textless{}} \FloatTok{0.05}\NormalTok{)) }\SpecialCharTok{+}
    \FunctionTok{geom\_point}\NormalTok{(}\AttributeTok{cex =} \FloatTok{2.5}\NormalTok{) }\SpecialCharTok{+}
    \FunctionTok{scale\_color\_manual}\NormalTok{(}\AttributeTok{values =} \FunctionTok{alpha}\NormalTok{(}\FunctionTok{c}\NormalTok{(}\StringTok{"black"}\NormalTok{, }\StringTok{"red"}\NormalTok{), }\FloatTok{0.5}\NormalTok{)) }\SpecialCharTok{+}
    \FunctionTok{theme\_minimal}\NormalTok{() }
\end{Highlighting}
\end{Shaded}

\includegraphics{pda_quantification_inference_files/figure-latex/unnamed-chunk-27-1.pdf}

A general class of moderated test statistics is given by \[
   T_g^{mod} = \frac{\bar{Y}_{g1} - \bar{Y}_{g2}}{C \quad \tilde{S}_g} ,
 \] where \(\tilde{S}_g\) is a moderated standard deviation estimate.

\begin{itemize}
\tightlist
\item
  \(C\) is a constant depending on the design
  e.g.~\(\sqrt{1/{n_1}+1/n_2}\) for a t-test and of another form for
  linear models.
\item
  \(\tilde{S}_g=S_g+S_0\): add small positive constant to denominator of
  t-statistic.
\item
  This can be adopted in Perseus.
\end{itemize}

Click to see code

\begin{Shaded}
\begin{Highlighting}[]
\NormalTok{simI}\OtherTok{\textless{}{-}}\FunctionTok{sapply}\NormalTok{(res}\SpecialCharTok{$}\NormalTok{se}\SpecialCharTok{/}\FunctionTok{sqrt}\NormalTok{(}\DecValTok{1}\SpecialCharTok{/}\DecValTok{3}\SpecialCharTok{+}\DecValTok{1}\SpecialCharTok{/}\DecValTok{3}\NormalTok{),}\ControlFlowTok{function}\NormalTok{(n,mean,sd) }\FunctionTok{rnorm}\NormalTok{(n,mean,sd),}\AttributeTok{n=}\DecValTok{6}\NormalTok{,}\AttributeTok{mean=}\DecValTok{0}\NormalTok{) }\SpecialCharTok{\%\textgreater{}\%}\NormalTok{ t}
\NormalTok{resSim }\OtherTok{\textless{}{-}} \FunctionTok{apply}\NormalTok{(}
\NormalTok{    simI, }
    \DecValTok{1}\NormalTok{, }
\NormalTok{    ttestMx,}
    \AttributeTok{group =} \FunctionTok{colData}\NormalTok{(pe)}\SpecialCharTok{$}\NormalTok{genotype}\SpecialCharTok{==}\StringTok{"D8"}\NormalTok{) }\SpecialCharTok{\%\textgreater{}\%} 
\NormalTok{  t }
 \FunctionTok{colnames}\NormalTok{(resSim) }\OtherTok{\textless{}{-}} \FunctionTok{c}\NormalTok{(}\StringTok{"logFC"}\NormalTok{,}\StringTok{"se"}\NormalTok{,}\StringTok{"tstat"}\NormalTok{,}\StringTok{"pval"}\NormalTok{)}
\NormalTok{ resSim }\OtherTok{\textless{}{-}} \FunctionTok{as.data.frame}\NormalTok{(resSim)}
\NormalTok{ tstatSimPlot }\OtherTok{\textless{}{-}}\NormalTok{ resSim }\SpecialCharTok{\%\textgreater{}\%} 
   \FunctionTok{ggplot}\NormalTok{(}\FunctionTok{aes}\NormalTok{(}\AttributeTok{x=}\NormalTok{tstat)) }\SpecialCharTok{+}
     \FunctionTok{geom\_histogram}\NormalTok{(}\FunctionTok{aes}\NormalTok{(}\AttributeTok{y=}\NormalTok{..density.., }\AttributeTok{fill=}\NormalTok{..count..),}\AttributeTok{bins=}\DecValTok{30}\NormalTok{) }\SpecialCharTok{+}
     \FunctionTok{stat\_function}\NormalTok{(}\AttributeTok{fun=}\NormalTok{dt,}
    \AttributeTok{color=}\StringTok{"red"}\NormalTok{,}
    \AttributeTok{args=}\FunctionTok{list}\NormalTok{(}\AttributeTok{df=}\DecValTok{4}\NormalTok{)) }\SpecialCharTok{+} 
   \FunctionTok{ylim}\NormalTok{(}\DecValTok{0}\NormalTok{,.}\DecValTok{6}\NormalTok{) }\SpecialCharTok{+}
   \FunctionTok{ggtitle}\NormalTok{(}\StringTok{"t{-}statistic"}\NormalTok{)}

 
\NormalTok{ resSim}\SpecialCharTok{$}\NormalTok{C }\OtherTok{\textless{}{-}} \FunctionTok{sqrt}\NormalTok{(}\DecValTok{1}\SpecialCharTok{/}\DecValTok{3}\SpecialCharTok{+}\DecValTok{1}\SpecialCharTok{/}\DecValTok{3}\NormalTok{) }
\NormalTok{ resSim}\SpecialCharTok{$}\NormalTok{sd }\OtherTok{\textless{}{-}}\NormalTok{ resSim}\SpecialCharTok{$}\NormalTok{se}\SpecialCharTok{/}\NormalTok{resSim}\SpecialCharTok{$}\NormalTok{C }
\NormalTok{ tstatSimPerseus }\OtherTok{\textless{}{-}}\NormalTok{ resSim }\SpecialCharTok{\%\textgreater{}\%} 
   \FunctionTok{ggplot}\NormalTok{(}\FunctionTok{aes}\NormalTok{(}\AttributeTok{x=}\NormalTok{logFC}\SpecialCharTok{/}\NormalTok{((sd}\FloatTok{+.1}\NormalTok{)}\SpecialCharTok{*}\NormalTok{C))) }\SpecialCharTok{+}
     \FunctionTok{geom\_histogram}\NormalTok{(}\FunctionTok{aes}\NormalTok{(}\AttributeTok{y=}\NormalTok{..density.., }\AttributeTok{fill=}\NormalTok{..count..),}\AttributeTok{bins=}\DecValTok{30}\NormalTok{) }\SpecialCharTok{+}
     \FunctionTok{stat\_function}\NormalTok{(}\AttributeTok{fun=}\NormalTok{dt,}
                   \AttributeTok{color=}\StringTok{"red"}\NormalTok{,}
                  \AttributeTok{args=}\FunctionTok{list}\NormalTok{(}\AttributeTok{df=}\DecValTok{4}\NormalTok{)) }\SpecialCharTok{+} 
     \FunctionTok{ylim}\NormalTok{(}\DecValTok{0}\NormalTok{,.}\DecValTok{6}\NormalTok{) }\SpecialCharTok{+}
    \FunctionTok{ggtitle}\NormalTok{(}\StringTok{"Persues"}\NormalTok{)}
\end{Highlighting}
\end{Shaded}

\begin{itemize}
\tightlist
\item
  The choice of \(S_0\) in Perseus is ad hoc and the t-statistic is
  no-longer t-distributed.
\item
  Permutation test, but is difficult for more complex designs.
\item
  Allows for Data Dredging because user can choose \(S_0\)
\end{itemize}

\hypertarget{empirical-bayes}{%
\subsubsection{Empirical Bayes}\label{empirical-bayes}}

A general class of moderated test statistics is given by \[
   T_g^{mod} = \frac{\bar{Y}_{g1} - \bar{Y}_{g2}}{C \quad \tilde{S}_g} ,
 \] where \(\tilde{S}_g\) is a moderated standard deviation estimate.

\begin{itemize}
\tightlist
\item
  \textbf{empirical Bayes} theory provides formal framework for
  borrowing strength across proteins,
\item
  Implemented in popular bioconductor package \textbf{limma} and
  \textbf{msqrob2}
\end{itemize}

\[
  \tilde{S}_g=\sqrt{\frac{d_gS_g^2+d_0S_0^2}{d_g+d_0}},
\] - \(S_0^2\): common variance (over all proteins) - Moderated
t-statistic is t-distributed with \(d_0+d_g\) degrees of freedom. - Note
that the degrees of freedom increase by borrowing strength across
proteins!

\begin{Shaded}
\begin{Highlighting}[]
\NormalTok{pe }\OtherTok{\textless{}{-}} \FunctionTok{msqrob}\NormalTok{(}\AttributeTok{object =}\NormalTok{ pe, }\AttributeTok{i =} \StringTok{"protein"}\NormalTok{, }\AttributeTok{formula =} \SpecialCharTok{\textasciitilde{}}\NormalTok{genotype)}
\NormalTok{L }\OtherTok{\textless{}{-}} \FunctionTok{makeContrast}\NormalTok{(}\StringTok{"genotypeD8 = 0"}\NormalTok{, }\AttributeTok{parameterNames =} \FunctionTok{c}\NormalTok{(}\StringTok{"genotypeD8"}\NormalTok{))}
\NormalTok{pe }\OtherTok{\textless{}{-}} \FunctionTok{hypothesisTest}\NormalTok{(}\AttributeTok{object =}\NormalTok{ pe, }\AttributeTok{i =} \StringTok{"protein"}\NormalTok{, }\AttributeTok{contrast =}\NormalTok{ L)}
\end{Highlighting}
\end{Shaded}

\begin{Shaded}
\begin{Highlighting}[]
\NormalTok{volcano }\OtherTok{\textless{}{-}} \FunctionTok{ggplot}\NormalTok{(}
    \FunctionTok{rowData}\NormalTok{(pe[[}\StringTok{"protein"}\NormalTok{]])}\SpecialCharTok{$}\NormalTok{genotypeD8,}
    \FunctionTok{aes}\NormalTok{(}\AttributeTok{x =}\NormalTok{ logFC, }\AttributeTok{y =} \SpecialCharTok{{-}}\FunctionTok{log10}\NormalTok{(pval), }\AttributeTok{color =}\NormalTok{ adjPval }\SpecialCharTok{\textless{}} \FloatTok{0.05}\NormalTok{)}
\NormalTok{) }\SpecialCharTok{+}
    \FunctionTok{geom\_point}\NormalTok{(}\AttributeTok{cex =} \FloatTok{2.5}\NormalTok{) }\SpecialCharTok{+}
    \FunctionTok{scale\_color\_manual}\NormalTok{(}\AttributeTok{values =} \FunctionTok{alpha}\NormalTok{(}\FunctionTok{c}\NormalTok{(}\StringTok{"black"}\NormalTok{, }\StringTok{"red"}\NormalTok{), }\FloatTok{0.5}\NormalTok{)) }\SpecialCharTok{+}
    \FunctionTok{theme\_minimal}\NormalTok{() }\SpecialCharTok{+}
    \FunctionTok{ggtitle}\NormalTok{(}\StringTok{"Default workflow"}\NormalTok{)}
\end{Highlighting}
\end{Shaded}

\begin{Shaded}
\begin{Highlighting}[]
\NormalTok{gridExtra}\SpecialCharTok{::}\FunctionTok{grid.arrange}\NormalTok{(volcanoT }\SpecialCharTok{+}    \FunctionTok{xlim}\NormalTok{(}\SpecialCharTok{{-}}\DecValTok{3}\NormalTok{,}\DecValTok{3}\NormalTok{)}
\NormalTok{,volcano }\SpecialCharTok{+}     \FunctionTok{xlim}\NormalTok{(}\SpecialCharTok{{-}}\DecValTok{3}\NormalTok{,}\DecValTok{3}\NormalTok{)}
\NormalTok{,}\AttributeTok{nrow=}\DecValTok{2}\NormalTok{)}
\end{Highlighting}
\end{Shaded}

\begin{verbatim}
## Warning: Removed 109 rows containing missing values (geom_point).
\end{verbatim}

\includegraphics{pda_quantification_inference_files/figure-latex/unnamed-chunk-31-1.pdf}

\hypertarget{shrinkage-of-the-variance-and-moderated-t-statistics}{%
\subsubsection{Shrinkage of the variance and moderated
t-statistics}\label{shrinkage-of-the-variance-and-moderated-t-statistics}}

\includegraphics{./figures/limmaShrinkage.png}

\begin{Shaded}
\begin{Highlighting}[]
\FunctionTok{qplot}\NormalTok{(}
  \FunctionTok{sapply}\NormalTok{(}\FunctionTok{rowData}\NormalTok{(pe[[}\StringTok{"protein"}\NormalTok{]])}\SpecialCharTok{$}\NormalTok{msqrobModels,getSigma),}
  \FunctionTok{sapply}\NormalTok{(}\FunctionTok{rowData}\NormalTok{(pe[[}\StringTok{"protein"}\NormalTok{]])}\SpecialCharTok{$}\NormalTok{msqrobModels,getSigmaPosterior)) }\SpecialCharTok{+}
  \FunctionTok{xlab}\NormalTok{(}\StringTok{"SD"}\NormalTok{) }\SpecialCharTok{+}
  \FunctionTok{ylab}\NormalTok{(}\StringTok{"moderated SD"}\NormalTok{) }\SpecialCharTok{+}
  \FunctionTok{geom\_abline}\NormalTok{(}\AttributeTok{intercept =} \DecValTok{0}\NormalTok{,}\AttributeTok{slope =} \DecValTok{1}\NormalTok{) }\SpecialCharTok{+}
  \FunctionTok{geom\_hline}\NormalTok{(}\AttributeTok{yintercept =}\NormalTok{ ) }
\end{Highlighting}
\end{Shaded}

\begin{verbatim}
## Warning: Removed 109 rows containing missing values (geom_point).
\end{verbatim}

\includegraphics{pda_quantification_inference_files/figure-latex/unnamed-chunk-33-1.pdf}

\hypertarget{plots}{%
\subsection{Plots}\label{plots}}

\begin{Shaded}
\begin{Highlighting}[]
\NormalTok{sigNames }\OtherTok{\textless{}{-}} \FunctionTok{rowData}\NormalTok{(pe[[}\StringTok{"protein"}\NormalTok{]])}\SpecialCharTok{$}\NormalTok{genotypeD8 }\SpecialCharTok{\%\textgreater{}\%}
    \FunctionTok{rownames\_to\_column}\NormalTok{(}\StringTok{"protein"}\NormalTok{) }\SpecialCharTok{\%\textgreater{}\%}
    \FunctionTok{filter}\NormalTok{(adjPval }\SpecialCharTok{\textless{}} \FloatTok{0.05}\NormalTok{) }\SpecialCharTok{\%\textgreater{}\%}
    \FunctionTok{pull}\NormalTok{(protein)}
\FunctionTok{heatmap}\NormalTok{(}\FunctionTok{assay}\NormalTok{(pe[[}\StringTok{"protein"}\NormalTok{]])[sigNames, ])}
\end{Highlighting}
\end{Shaded}

\includegraphics{pda_quantification_inference_files/figure-latex/unnamed-chunk-34-1.pdf}

\begin{Shaded}
\begin{Highlighting}[]
\ControlFlowTok{for}\NormalTok{ (protName }\ControlFlowTok{in}\NormalTok{ sigNames)}
\NormalTok{    \{}
\NormalTok{        pePlot }\OtherTok{\textless{}{-}}\NormalTok{ pe[protName, , }\FunctionTok{c}\NormalTok{(}\StringTok{"peptideNorm"}\NormalTok{, }\StringTok{"protein"}\NormalTok{)]}
\NormalTok{        pePlotDf }\OtherTok{\textless{}{-}} \FunctionTok{data.frame}\NormalTok{(}\FunctionTok{longFormat}\NormalTok{(pePlot))}
\NormalTok{        pePlotDf}\SpecialCharTok{$}\NormalTok{assay }\OtherTok{\textless{}{-}} \FunctionTok{factor}\NormalTok{(pePlotDf}\SpecialCharTok{$}\NormalTok{assay,}
            \AttributeTok{levels =} \FunctionTok{c}\NormalTok{(}\StringTok{"peptideNorm"}\NormalTok{, }\StringTok{"protein"}\NormalTok{)}
\NormalTok{        )}
\NormalTok{        pePlotDf}\SpecialCharTok{$}\NormalTok{genotype }\OtherTok{\textless{}{-}} \FunctionTok{as.factor}\NormalTok{(}\FunctionTok{colData}\NormalTok{(pePlot)[pePlotDf}\SpecialCharTok{$}\NormalTok{colname, }\StringTok{"genotype"}\NormalTok{])}

        \CommentTok{\# plotting}
\NormalTok{        p1 }\OtherTok{\textless{}{-}} \FunctionTok{ggplot}\NormalTok{(}
            \AttributeTok{data =}\NormalTok{ pePlotDf,}
            \FunctionTok{aes}\NormalTok{(}\AttributeTok{x =}\NormalTok{ colname, }\AttributeTok{y =}\NormalTok{ value, }\AttributeTok{group =}\NormalTok{ rowname)}
\NormalTok{        ) }\SpecialCharTok{+}
            \FunctionTok{geom\_line}\NormalTok{() }\SpecialCharTok{+}
            \FunctionTok{geom\_point}\NormalTok{() }\SpecialCharTok{+}
            \FunctionTok{theme\_minimal}\NormalTok{() }\SpecialCharTok{+}
            \FunctionTok{facet\_grid}\NormalTok{(}\SpecialCharTok{\textasciitilde{}}\NormalTok{assay) }\SpecialCharTok{+}
            \FunctionTok{ggtitle}\NormalTok{(protName)}
        \FunctionTok{print}\NormalTok{(p1)}

        \CommentTok{\# plotting 2}
\NormalTok{        p2 }\OtherTok{\textless{}{-}} \FunctionTok{ggplot}\NormalTok{(pePlotDf, }\FunctionTok{aes}\NormalTok{(}\AttributeTok{x =}\NormalTok{ colname, }\AttributeTok{y =}\NormalTok{ value, }\AttributeTok{fill =}\NormalTok{ genotype)) }\SpecialCharTok{+}
            \FunctionTok{geom\_boxplot}\NormalTok{(}\AttributeTok{outlier.shape =} \ConstantTok{NA}\NormalTok{) }\SpecialCharTok{+}
            \FunctionTok{geom\_point}\NormalTok{(}
                \AttributeTok{position =} \FunctionTok{position\_jitter}\NormalTok{(}\AttributeTok{width =}\NormalTok{ .}\DecValTok{1}\NormalTok{),}
                \FunctionTok{aes}\NormalTok{(}\AttributeTok{shape =}\NormalTok{ rowname)}
\NormalTok{            ) }\SpecialCharTok{+}
            \FunctionTok{scale\_shape\_manual}\NormalTok{(}\AttributeTok{values =} \DecValTok{1}\SpecialCharTok{:}\FunctionTok{nrow}\NormalTok{(pePlotDf)) }\SpecialCharTok{+}
            \FunctionTok{labs}\NormalTok{(}\AttributeTok{title =}\NormalTok{ protName, }\AttributeTok{x =} \StringTok{"sample"}\NormalTok{, }\AttributeTok{y =} \StringTok{"peptide intensity (log2)"}\NormalTok{) }\SpecialCharTok{+}
            \FunctionTok{theme\_minimal}\NormalTok{()}
        \FunctionTok{facet\_grid}\NormalTok{(}\SpecialCharTok{\textasciitilde{}}\NormalTok{assay)}
        \FunctionTok{print}\NormalTok{(p2)}
\NormalTok{\}}
\end{Highlighting}
\end{Shaded}

\includegraphics{pda_quantification_inference_files/figure-latex/unnamed-chunk-35-1.pdf}
\includegraphics{pda_quantification_inference_files/figure-latex/unnamed-chunk-35-2.pdf}
\includegraphics{pda_quantification_inference_files/figure-latex/unnamed-chunk-35-3.pdf}
\includegraphics{pda_quantification_inference_files/figure-latex/unnamed-chunk-35-4.pdf}
\includegraphics{pda_quantification_inference_files/figure-latex/unnamed-chunk-35-5.pdf}
\includegraphics{pda_quantification_inference_files/figure-latex/unnamed-chunk-35-6.pdf}
\includegraphics{pda_quantification_inference_files/figure-latex/unnamed-chunk-35-7.pdf}
\includegraphics{pda_quantification_inference_files/figure-latex/unnamed-chunk-35-8.pdf}
\includegraphics{pda_quantification_inference_files/figure-latex/unnamed-chunk-35-9.pdf}
\includegraphics{pda_quantification_inference_files/figure-latex/unnamed-chunk-35-10.pdf}
\includegraphics{pda_quantification_inference_files/figure-latex/unnamed-chunk-35-11.pdf}
\includegraphics{pda_quantification_inference_files/figure-latex/unnamed-chunk-35-12.pdf}
\includegraphics{pda_quantification_inference_files/figure-latex/unnamed-chunk-35-13.pdf}
\includegraphics{pda_quantification_inference_files/figure-latex/unnamed-chunk-35-14.pdf}
\includegraphics{pda_quantification_inference_files/figure-latex/unnamed-chunk-35-15.pdf}
\includegraphics{pda_quantification_inference_files/figure-latex/unnamed-chunk-35-16.pdf}
\includegraphics{pda_quantification_inference_files/figure-latex/unnamed-chunk-35-17.pdf}
\includegraphics{pda_quantification_inference_files/figure-latex/unnamed-chunk-35-18.pdf}
\includegraphics{pda_quantification_inference_files/figure-latex/unnamed-chunk-35-19.pdf}
\includegraphics{pda_quantification_inference_files/figure-latex/unnamed-chunk-35-20.pdf}
\includegraphics{pda_quantification_inference_files/figure-latex/unnamed-chunk-35-21.pdf}
\includegraphics{pda_quantification_inference_files/figure-latex/unnamed-chunk-35-22.pdf}
\includegraphics{pda_quantification_inference_files/figure-latex/unnamed-chunk-35-23.pdf}
\includegraphics{pda_quantification_inference_files/figure-latex/unnamed-chunk-35-24.pdf}

\hypertarget{experimental-design}{%
\section{Experimental Design}\label{experimental-design}}

\hypertarget{sample-size}{%
\subsection{Sample size}\label{sample-size}}

\[
 \log_2 \text{FC} = \bar{y}_{p1}-\bar{y}_{p2}
\]

\[
T_g=\frac{\log_2 \text{FC}}{\text{se}_{\log_2 \text{FC}}}
\]

\[
T_g=\frac{\widehat{\text{signal}}}{\widehat{\text{Noise}}}
\]

If we can assume equal variance in both treatment groups:

\[
\text{se}_{\log_2 \text{FC}}=\text{SD}\sqrt{\frac{1}{n_1}+\frac{1}{n_2}}
\]

\(\rightarrow\) if number of bio-repeats increases we have a higher
power!

\begin{itemize}
\tightlist
\item
  cfr. Study of tamoxifen treated Estrogen Recepter (ER) positive breast
  cancer patients
\end{itemize}

\hypertarget{blocking}{%
\subsection{Blocking}\label{blocking}}

\[\sigma^2= \sigma^2_{bio}+\sigma^2_\text{lab} +\sigma^2_\text{extraction} + \sigma^2_\text{run} + \ldots\]

\begin{itemize}
\tightlist
\item
  Biological: fluctuations in protein level between mice, fluctations in
  protein level between cells, \ldots{}
\item
  Technical: cage effect, lab effect, week effect, plasma extraction,
  MS-run, \ldots{}
\end{itemize}

\hypertarget{nature-methods-points-of-significance---blocking}{%
\subsection{Nature methods: Points of significance -
Blocking}\label{nature-methods-points-of-significance---blocking}}

\url{https://www.nature.com/articles/nmeth.3005.pdf}

\hypertarget{mouse-example}{%
\subsection{Mouse example}\label{mouse-example}}

\includegraphics[width=0.5\linewidth]{./figures/mouseTcell_RCB_design}

\begin{itemize}
\tightlist
\item
  All treatments of interest are present within block!
\item
  We can estimate the effect of the treatment within block!
\item
  We can isolate the between block variability from the analysis using
  linear model: \[ 
  y \sim \text{type} + \text{mouse}
  \]
\item
  Not possible with Perseus!
\end{itemize}

\hypertarget{assess-the-impact-of-blocking-in-the-tutorial-session}{%
\subsubsection{Assess the impact of blocking in the tutorial
session!}\label{assess-the-impact-of-blocking-in-the-tutorial-session}}

\begin{itemize}
\tightlist
\item
  Completely randomized design with only one cell type per mouse (Treg
  and Tconv)
\end{itemize}

\[\updownarrow\]

\begin{itemize}
\tightlist
\item
  Randomized complete block design assessing Treg and Tconv on each
  mouse
\end{itemize}

\end{document}
